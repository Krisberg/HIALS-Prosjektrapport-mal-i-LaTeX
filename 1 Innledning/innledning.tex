\documentclass[../main.tex]{subfiles} 
\begin{document}

\clearpage\section{INNLEDNING}

\bigskip

{\itshape\color{blue}
[Dette er f{\o}rste kapitlet i den faglige rapporten. \ Det b{\o}r behandle bakgrunnen for oppgaven, oppdragsgiver,
problemstillingen med problemets historikk og/eller oppgaven som skal l{\o}ses. Her skal du ogs{\aa} si noe om omfanget
eller avgrensningen av oppgaven.}

{\itshape\color{blue}
Til slutt skal du kort beskrive hva rapporten videre inneholder, m.a.o. hva kan leseren forvente {\aa} lese om i
rapporten.}

{\itshape\color{blue}
\textbf{\textup{Kommentar}}: Det er her dere skal gi en innledning eller en slags presentasjon av hele oppgaven. Og det
er ogs{\aa} her dere skal presentere selve problemstillingen som skal l{\o}ses og eventuelle avgrensninger som
gj{\o}res. Merk at det er p{\aa} denne problemstillingen som resultatdelen og konklusjonen skal vise en l{\o}sning for.
}

{\itshape\color{blue}
Dersom oppgaven har f{\aa}tt utdelt en kravspesifikasjon, skal hovedtrekkene fra kravspesifikasjonen skisseres her, med
referanse til den fullstendige kravspesifikasjonen.]}


\bigskip

\end{document}