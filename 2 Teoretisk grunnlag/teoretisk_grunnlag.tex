\documentclass[../main.tex]{subfiles} 
\begin{document}

\section{TEORETISK GRUNNLAG}

\bigskip

{\itshape\color{blue}
[Oppgaver og problemstillinger st{\aa}r i en sammenheng. Denne delen skal vise at en har oversikt over denne
sammenhengen, at en er eller har gjort seg kjent med tidligere resultater og andres forslag til eller fors{\o}k p{\aa}
l{\o}sninger. Det er alts{\aa} tale om {\aa} gi et faglig underlag for ens eget arbeid, evt. en beskrivelse av
teoretiske forutsetninger, med referanse til litteratur og andre kilder en st{\o}tter seg til.}

{\itshape\color{blue}
Hovedhensikten med avsnittet er {\aa} vise og dokumentere at ditt prosjekt/din besvarelse er underst{\o}ttet av
tilsvarende eller tilst{\o}tende forskning, og ikke bare noe du har kommet p{\aa} selv uten forankring i eksisterende
teorier/anbefalinger.}

{\bfseries\itshape\color{blue}
Kommentar }

{\itshape\color{blue}
\textbf{Generellt:} Det kan v{\ae}re naturlig {\aa} presentere den teori som er relevant for de vurderinger som skal gi
en l{\o}sning p{\aa} problemstillingen Det viktige her er {\aa} f{\aa} fram det teoretiske grunnlaget dere senere skal
bruke til {\aa} vurdere og argumentere for en foresl{\aa}tt l{\o}sning. }

{\itshape\color{blue}
Alle vurderinger dere gj{\o}r senere i besvarelsen skal ha referanse til dette kapittelet. Det er s{\ae}rdeles viktig
{\aa} ha tydelige referanser til de kildene dere bruker n{\aa}r dere skriver dette kapittelet. All teori dere beskriver
her skal alts{\aa} ha en referanse, og denne skal skrives inn i teksten}

{\itshape\color{blue}
\textbf{I typiske SW utviklingsprosjekter:} I SW prosjekter, er fokuset i tillegg til {\aa} oppfylle et sett av
funksjonelle krav, ofte {\aa} utvikle systemet i henhold til anbefalte og gode prinsipper og teorier for modul{\ae}re,
objektorienterte systemer. Det vil da v{\ae}re viktig {\aa} dokumentere i dette avsnittet hvilke designprinsipper man
har lagt til grunn. Stikkord her kan v{\ae}re cohesion, coupling, design patterns, arkitektur patterns osv. Det finnes
ogs{\aa} en del teorier rundt hvordan man b{\o}r gjennomf{\o}re et SW-prosjekt, som for eksempel iterativ utvikling, XP
(extreeme programming), Agile processes etc. som man har benyttet til {\aa} l{\o}se problemstillingen.}

{\bfseries\itshape\color{blue}
Eksempel:}

{\itshape\color{blue}
{}''{\dots}et viktig moment ved analysen av problemstillingen er {\aa} identifisere kandidater til objekter som senere
danner grunnlag for klassene. En mye benyttet metode i f{\o}lge l{\ae}reboka [1]{\dots}..etc.''}


\bigskip

{\itshape\color{blue}
\textbf{I automasjonsprosjekter:} I slike prosjekter er det ofte snakk om l{\o}sninger der man skal utf{\o}re en del
beregninger basert p{\aa} grunnlagsdata eller innsamlede data. I slike prosjekter skal man i dette avsnittet presentere
de teorier som man har benyttet seg av i prosjektet. Det skal gis et kort sammendrag av teoriene med eventuelle
tilh{\o}rende matematiske formler, med referanse til kilder der teorien er beskrevet utf{\o}rlig (b{\o}ker,
publikasjoner, tidsskrifter etc.)]}


\bigskip

\end{document}